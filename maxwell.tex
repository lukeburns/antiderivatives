\documentclass{article}
\usepackage[margin=1in]{geometry}
\usepackage{amsmath}
% \usepackage{hyperref}
% \hypersetup{colorlinks=true}

\title{The Continuity Equation Implies Maxwell's Equations}
\author{Luke Burns}

\begin{document}
\maketitle
\abstract{It is shown that the antiderivative of a coclosed (closed) multivector field fails to be closed (coclosed) by at most a harmonic term, from which it follows that \emph{any} vector valued current density $J$ in four dimensions that is conserved (i.e. is coclosed) possesses a bivector valued antiderivative $F$ that satisfies Maxwell's equations $\partial F = J$ under physically reasonable boundary conditions.}

\section{Introduction}

After establishing a mapping between $n$-vector fields and differential forms of degree $n$, I present two key results of geometric calculus: a generalized Integral Formula and Helmholtz decomposition for fields. 

Using these results, I show that all closed fields are inexact by at most a monogenic term. Monogenic fields are characterized by the property that they are fully determined by boundary conditions, analogous to complex analytic functions. I present some conditions under which these fields are exact. A field whose antiderivative is closed is dubbed \emph{faithful}, by which it follows that \emph{``the derivative of a closed field is faithful, and the antiderivative of a faithful field is closed.''}

This establishes an equivalence between the statements \emph{``an electromagnetic field $F$ is a closed bivector''} and \emph{``an electromagnetic current $J$ is a faithful vector,''} both of which fully determine the structure of Maxwell's equations. I then show that any conserved vector valued field $J$ is faithful under physically reasonable boundary conditions.

\section{Fields and forms} If $F_n \equiv \langle F \rangle_n$ is the grade $n$ part of the multivector field (hereafter, just field) $F$, then its corresponding \emph{differential form $f_n$ of degree $n$} is a scalar field given by \cite{cagc}

\begin{equation}
  f_n \equiv d^nx^\dagger \cdot F_n,\label{eq:form}
\end{equation}

which is the projection of the $n$-vector field $F_n$ onto the directed measure $d^nx^\dagger = dx_n \wedge \dots \wedge dx_1$, where $dx_i$ are vector valued differentials. 

The hodge star operation $*$ acts on fields as

\begin{equation}
  *F \equiv F^\dagger I,\label{eq:hodge}
\end{equation}

where $I$ is the pseudoscalar of some oriented vector manifold.

The exterior derivative $d$ behaves identically to the curl

\begin{equation}
  d f_n \equiv d^{n+1}x^\dagger \cdot (\partial \wedge F_n),\label{eq:curl}
\end{equation}

and the ``adjoint operator'' $\delta$ behaves identically to (minus) the divergence

\begin{equation}
  \delta f_n \equiv d^{n-1}x^\dagger \cdot (-\partial \cdot F_n).\label{eq:divergence}
\end{equation} 

The word \emph{form} will be reserved for scalar fields corresponding to some $n$-vector field via Equation \ref{eq:form}. Lowercase letters will be used for forms and uppercase letters for fields. Subscripts denote grade (degree).

\section{Derivatives}

A field $F$ is called \emph{closed or curl free} when 

\begin{equation}
  \partial \wedge F = 0
\end{equation}

and \emph{coclosed or divergence free} when 

\begin{equation}
  \partial \cdot F = 0.
\end{equation}

A field with no divergence or curl (closed and coclosed)

\begin{equation}
  \partial F = \partial \cdot F + \partial \wedge F = 0
\end{equation}

is called \emph{monogenic}. It possesses the property of complex analytic functions that, in any region, it is fully determined by its values on the boundary of that region. Hence, the form $\omega$ is closed if $d \omega = 0$, coclosed if $\delta \omega = 0$, and monogenic if $d \omega = \delta \omega = 0$.

A field $H$ that satisfies

\begin{equation}
  \partial^2 H = 0
\end{equation}

might be called \emph{harmonic}, although the term is inappropriate in mixed signature spaces. For instance, in Minkowski space, $\partial^2 H = (\partial_t^2 - \vec \nabla^2) H = 0$ is the wave equation and its properties are differ dramatically from the usual harmonic functions in Euclidean spaces. Nonetheless, we will abuse the term here for lack of a better one. A form $\gamma$ is then harmonic if $d \delta \gamma + \delta d \gamma = 0$.

\section{Potentials}

If a field $J$ is written as

\begin{equation}
  J = \partial \cdot G + \partial \wedge H,
\end{equation}

then $G$ and $H$ are called \emph{potentials} for $J$. By extension, if a form $\omega$ is given by

\begin{equation}
  \omega = d \alpha + \delta \beta
\end{equation}

then $\alpha$ and $\beta$ will be called potentials for $\omega$.

A field $J$ is called \emph{exact} when

\begin{equation}
  J = \partial \wedge F
\end{equation}

and \emph{coexact} when

\begin{equation}
  J = \partial \cdot F,
\end{equation}

whereby a form $\omega$ is exact if $\omega = d \alpha$ and coexact if $\omega = \delta \beta$.

\section{Antiderivatives}

A field $F$ is called an \emph{antiderivative} of $J$ if

\begin{equation}
  J = \partial F = \partial \cdot F + \partial \wedge F,\label{eq:antiderivative}
\end{equation}

which is unique up to a monogenic term. That is, $F + C$ such that $\partial C = 0$ is also an antiderivative. Furthermore, given an antiderivative, one has possession of constraints on $F$. For every $J_k = 0$,

\begin{equation}
  J_k = \partial \cdot F_{k+1} + \partial \wedge F_{k-1} = 0.\label{eq:constraints}
\end{equation}

As an example, if $J = J_n$ is an $n$-vector field, then 

\begin{equation}
  J_n = \partial F = \partial \cdot F_{n+1} + \partial \wedge F_{n-1},\label{ex:antiderivative}
\end{equation}

and the constraints due to $J_{n-1} = J_{n+1} = 0$ are

\begin{equation}
  \partial \cdot F_{n-1} = \partial \wedge F_{n+1} = 0.\label{ex:constraints}
\end{equation}

Of course, $F$ \emph{could} contain terms of higher and lower grades, but they make no contribution to $J_n$. In this case, it will generally be of the form $F = F_{n-1} + F_{n+1} + C$.

If $j_n$ and $f_n$ are the forms given by $J_n$ and $F_n$, then Equation \ref{ex:antiderivative} is equivalent to

\begin{equation}
  j_n = \delta f_{n+1} + d f_{n-1},
\end{equation}

and Equation \ref{ex:constraints} is equivalent to

\begin{equation}
  \delta f_{n-1} = d f_{n+1} = 0.
\end{equation}

Given potentials $f_{n-1}$ and $f_{n+1}$ under these constraints, one is in possession of an antiderivative of $j_n$.

\section{The Fundamental Theorem}

Let $\mathcal{M}$ be an m-dimensional smooth oriented vector manifold with a piecewise smooth boundary $\partial \mathcal{M}$ and $L$ be a linear function, differentiable on $\mathcal{M}$ and $\partial \mathcal{M}$. Then, \cite{cagc}

\begin{equation}
  \int L(\dot x, d^mx \cdot \dot \partial) = \oint L(x, d^{m-1}x),\label{eq:fundamental-theorem}
\end{equation}

where $L(\dot x, d^mx \cdot \dot \partial)$ denotes right and left differentiation all $x$ dependent terms by $\partial$.

Stokes' theorem of differential forms is

\begin{equation}
  \int \langle L(\dot x, d^mx \cdot \dot \partial)\rangle = \oint \langle L(x, d^{m-1}x)\rangle,
\end{equation}

for scalar valued integrands.

\section{Integral Formula}

Let $J$ be a field on a simple (not self-intersecting) manifold $\mathcal{M}$ subject to the same criteria in the fundamental theorem. Suppose $J$ satisfies the equation

\begin{equation}
  \partial F = J.
\end{equation}

Then $F$ is given by \cite{cagc}

\begin{equation}
  F(x) = (-1)^m I^{-1}(x) \left(\int g(x, x') d^{m}x' J(x') - \oint g(x, x') d^{m-1}x' F(x')\right),\label{eq:integral-formula}
\end{equation}

where $g$ is a Green's function of $\partial$ satisfying $\partial g(x,x') = - g(x, x') \partial' = \delta(x - x').$ This result says that \emph{any integrable field has an antiderivative, and it's given by Equation \ref{eq:integral-formula}}.

\section{Helmholtz decomposition}

The integral forumla tells us that $J$ has an antiderivative $F$ such that

\begin{align}
  J = \partial F = \partial \cdot F + \partial \wedge F &= (-1)^m I^{-1} \left(\int g d^{m}x \cdot \partial^2 F - \oint g d^{m-1}x \partial F\right).\label{eq:helmholtz}
\end{align}

In addition, we can say

\begin{equation}
  \partial \cdot F = (-1)^m I^{-1} \left(\int g d^{m}x \cdot \partial (\partial \cdot F) - \oint g d^{m-1}x \partial \cdot F\right)
\end{equation}

and

\begin{equation}
  \partial \wedge F = (-1)^m I^{-1} \left(\int g d^{m}x \cdot \partial (\partial \wedge F) - \oint g d^{m-1}x \partial \wedge F\right),
\end{equation}

which gives a generalized Helmholtz decomposition into divergence free (or coclosed) and curl free (or closed) fields, $\partial \cdot F$ and $\partial \wedge F$ respectively. This is because $\partial \wedge (\partial \wedge M) = \partial \cdot (\partial \cdot M) = 0$ for any field $M$. 

Additionally, this decomposition comes with constraints given by Equation \ref{eq:constraints}.

\section{Antiderivatives of divergence free (or coclosed) fields} The above result implies that antiderivatives of divergence free fields fail to be curl free, and antiderivatives of curl free fields fail to be divergence free, by at most a harmonic term $H$ satisfying $\partial^2 H = 0$.

Suppose $J = \partial F$ is divergence free (the dual result for curl free fields follows identically). Then, 

\begin{equation}
  \partial \cdot J = \partial \cdot (\partial F) = \partial \cdot (\partial \wedge F) = \partial (\partial \wedge F) = 0,
\end{equation}

which means that $C \equiv \partial \wedge F$ is monogenic and $J$ is \emph{cohomologous} with $C$

\begin{equation}
  J - C = \partial F - C = \partial \cdot F,\label{eq:cohomologous}
\end{equation}

because their difference is coexact.

Employing the integral theorem, $C$ has an antiderivative $H$ such that 

\begin{equation}
  C = \partial H.
\end{equation}

With $G \equiv F - H$, this implies that $F$ can then be written

\begin{equation}
  F = G + H\label{eq:harmonic-split}
\end{equation}

where $\partial G = \partial \cdot F$ and $\partial^2 H = 0$. Hence, $F$ fails to be closed by at most a harmonic term $H$.

As an example, if $C$ is an r-vector field, then $C$ can be written $C = \partial \cdot (x \wedge C)/r = \partial \wedge (x \cdot C)/(n-r)$, and $C$ is both exact and coexact, in which case\footnote{Under what conditions are monogenic fields (co)exact?}

\begin{equation}
  J = \partial F = \partial \cdot (F + x \wedge C/r)
\end{equation}

is coexact --- although, $F$ is not closed. 

If $C = 0$ on the boundary, then $C = 0$ everywhere, and its antiderivative is closed

\begin{equation}
  J = \partial F = \partial \cdot F. \label{eq:faithful}
\end{equation}

Let us call a field $J$ \emph{faithful} if its antiderivative is closed. Faithful fields are coexact, and all coclosed fields differ from faithful fields by at most a monogenic field, which depends solely on the manifold and boundary conditions. Note, however, that coexact fields are not necessarily faithful.

\section{Maxwell's equations}

Maxwell's equations follow from directly from the statement that \emph{``an electromagnetic field $F$ is a closed bivector field and its derivative is its current $J$,''} which means that

\begin{equation}
  \partial \wedge F = 0.
\end{equation}

This implies that its derivative $J$ is

\begin{equation}
  J = \partial F = \partial \cdot F,
\end{equation} 

which are Maxwell's equations with no magnetic monopoles or currents. As is well known, Maxwell's equations imply the continuity equation

\begin{equation}
  \partial \cdot J = \partial \cdot (\partial \cdot F) = 0, \label{eq:continuity}
\end{equation} 

which means charge is conserved. $F$ is the antiderivative of $J$, so to say that $F$ is closed is the same as saying $J$ is faithful. Hence, the above is equivalent to a dual statement \emph{``an electromagnetic current $J$ is a faithful vector field and its antiderivative is its electromagnetic field $F$.''}

Suppose $J$ is a conserved current. Then by Equation \ref{ex:constraints} and Equation \ref{eq:harmonic-split}, its antiderivative $F$ can be decomposed into 

\begin{equation}
  F = F_0 + F_2,
\end{equation}

where $F_0$ is harmonic. 

Equation \ref{eq:helmholtz} tells us that, in four dimensions, $J$ can be written

\begin{align}
  J = \partial F &= I^{-1} \left(\int g d^{4}x \cdot \partial^2 F - \oint g d^{3}x J \right).
\end{align}

If we integrate over all of spacetime, the boundary term vanishes for reasonable charge distributions, and $J$ is given by

\begin{align}
  J = \partial F &= I^{-1} \int g d^{4}x \cdot \partial^2 F,
\end{align}

which is only dependent on $\partial^2 F = \partial^2 (F_0 + F_2) = \partial^2 F_2$, since $F_0$ is harmonic. Hence, $J$ is independent of $F_0$, and $J$ can at last be written as 

\begin{equation}
  J = \nabla F = \nabla \cdot F,
\end{equation}

and $J$ is faithful.

% Hence, if $J$ is conserved,

% \begin{equation}
%   J = \partial F = \partial \cdot F
% \end{equation}

% and

% \begin{equation}
%   \partial \wedge F = (-1)^{m} I^{-1} \oint g d^{m-1}x \partial \cdot F\right),
% \end{equation}


  \begin{thebibliography}{9} 

    \bibitem{cagc}
      D. Hestenes.
      \emph{Clifford algebra to geometric calculus}.
      D. Reidel Publishing Company (1984).

    % \bibitem{gap} 
    %   C. Doran and A. Lasenby.
    %   \emph{Geometric Algebra for Physicists}. Cambridge University Press (2003).

  \end{thebibliography}

\end{document}
