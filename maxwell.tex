\documentclass{article}
\usepackage[margin=1in]{geometry}
\usepackage{amsmath}
\usepackage{sectsty}
\usepackage{amsfonts}
\usepackage{multicol}
\usepackage{hyperref}

\sectionfont{\fontsize{12}{15}\selectfont}
\renewcommand{\abstractname}{\vspace{-\baselineskip}}

\title{When does the continuity equation imply Maxwell's equations?}
\author{Luke Burns}
\date{\small \today\vspace{-5ex}}

\begin{document}

\maketitle

\abstract{A fundamental result of electrodynamics is that Maxwell's equations imply that charge is locally conserved. Here we show that local charge conservation implies the existence of fields satisfying Maxwell's equations.}

\section{Introduction}

In Fredriech Hehl's premetric approach to electrodynamics, the inhomogenous Maxwell equations $\nabla \cdot F = J$ are obtained from the continuity equation $\nabla \cdot J = 0$ using Poincar\'e's lemma, and the homogenous equations $\nabla \wedge F = 0$ are determined by a separate postulate.\cite{hehl} \footnote{More precisely, Hehl postulates the existence of a field strength $F$ and excitation $H$ related by some constitutive relations satisfying $\nabla \wedge F = 0$ and $\nabla \cdot H = J$. While the use of excitations and constitutive relations are surely at the crux of the premetric approach, it can be argued, as Griffiths does, that the division into field strength and excitation ``reflect[s] a convenient division of charge and current into free and nonfree parts,'' so in this way the ``microscopic'' equations are not any less general if we've already established a metric. See p330 of \cite{griffiths}.} 

In this paper, we present a strengthened version of the Poincar\'e Lemma that has as an immediate consequence that local charge conservation is enough to determine the existence of a field satisfying both.

This fact allows us to understand Maxwell's equation precisely as an expression of local charge conservation, deepens our understanding of analogies to electrodynamics found across theories, and can help guide investigations into extensions of electrodynamics, as well as the construction of other theories with conserved quantities.

We begin with a statement of the key result of geometric calculus used (the integral formula), follow with the Poincar\'e lemma, and then present its stronger form. We end with its application answering the question posed in the title and a brief discussion.

\section{The Integral Formula}

Let $J$ be a multivector field integrable on a simple, m-dimensional manifold $\mathcal{M}$. Then it possesses antiderivatives $H$ (i.e. $J = \nabla H$) determined by $J$, up to boundary conditions, given by

\begin{align}
  \begin{split}
    (-1)^m i(x) H(x) = - \int g(x, x') d^m x' J(x') + \oint g(x, x') d^{m-1} x' H(x'),\label{eq:integral-formula}
  \end{split}
\end{align}

where $i$ is the unit pseudoscalar, $g$ is a Green's function defined on $\mathcal{M}$ (granted that one exists), $\nabla$ is the vector derivative, and $d^k x$ is a $k$-graded differential. Note that antiderivatives differ at most by a monogenic function $C$, i.e. one that satisfies $\nabla C = 0$. See p260 of \cite{cagc} for a complete development. 

There are subtleties around the surfaces used in a mixed signature space. In particular, the boundary cannot be a null surface. See p182 of \cite{gap} for a detailed discussion.

Note that this result generalizes Helmholtz decomposition of vector calculus to arbitrary dimension, as it implies that any integrable multivector field $A = \nabla B$ with antiderivative $B$ can be decomposed into the sum of a divergence-free and curl-free part

\begin{align}
  A = A^\perp + A^\parallel, \label{eq:helmholtz}
\end{align}

with $A^\perp = \nabla \cdot B$ and $A^\parallel = \nabla \wedge B.$\footnote{This is due to the identities $\nabla \cdot (\nabla \cdot M) = 0$ and $\nabla \wedge (\nabla \wedge M) = 0$ for any multivector field $M$.} We'll use this notation $A^\perp$ and $A^\parallel$ to indicate divergence-free and curl-free parts for any multivector field $A$, respectively, extending a notation used by Belinfante. \cite{belinfante}

\section{A strengthening of Poincar\'e's lemma}

In the language of differential forms, the Poincar\'e lemma states that closed differential forms are locally exact. In geometric calculus, every $k$-form corresponds to a $k$-vector field, and the Poincar\'e lemma can be restated as: every curl free k-vector field can be written as the curl of a $(k-1)$-vector field locally. See \cite{hestenes} for the embedding of differential forms in geometric calculus used here. 

More generally, for any multivector $J$ on $\mathcal{M}$, 

\begin{equation}
    \nabla \wedge J = 0 \implies \exists G : J = \nabla \wedge G
\end{equation} 

for some multivector field $G$, locally.\footnote{The Poincar\'e lemma extends to general multivectors, because it can simply be applied component-wise: $\nabla \wedge J = 0$ implies $\nabla \wedge \langle J \rangle_k = 0$ for each k-grade part of $J$. This implies $\langle J \rangle_k = \nabla \wedge G_{k-1}$ so that $J = \nabla \wedge G$, where $\langle G \rangle_k = G_k$.} The dual of this result is

\begin{equation}
    \nabla \cdot J = 0 \implies \exists G : J = \nabla \cdot G,
\end{equation}

for some multivector $G$. Again, that this holds only locally is crucial as the failure of it to do so provides a measure of the manifold's homology via the de Rham cohomology.

A stronger version of Poincar\'e's lemma can be obtained by pairing it with Helmholtz decomposition, Equation~\ref{eq:helmholtz}.

Namely, if $J$ is curl free, then there exists a multivector $G$ such that

\begin{equation}
    J = \nabla \wedge G = \nabla \wedge (G^\perp + G^\parallel) = \nabla \wedge G^\perp = \nabla G^\perp.
\end{equation}

That is, if $J$ is curl free, then locally it possesses a divergence free antiderivative $G^\perp.$\footnote{In the language of differential forms, this ``strong'' Poincare lemma can be expressed as ${d \alpha = 0 \implies \alpha = d \beta \text{, with } d * \beta = 0}$, under the usual condition.}

The dual also holds. If $J$ is divergence free, then there exists a $G$ such that

\begin{equation}
    J = \nabla \cdot G = \nabla \cdot (G^\perp + G^\parallel) = \nabla \cdot G^\parallel = \nabla G^\parallel.
\end{equation}

That is, if $J$ is divergence free, then locally it possesses a curl free antiderivative $G^\parallel.$ 

Curiously, to my knowledge, this result hasn't much found footing in the literature, despite the fact that it provides an immediate answer to the title of this paper. Although, one instance of this theorem in Euclidean space can be found in \cite{brackx} by Brackx, Delanghe, and Sommen, who also noted its seeming absence from the literature.

\section{Maxwell's equation}

Turning to the question ``When does the continuity equation imply Maxwell's equation?'', simply notice that the continuity equation $\nabla \cdot J = 0$ expresses the fact that $J$ (here a vector field) is divergence free, which implies that there exists some multivector field $F$ satisfying

\begin{equation}
    J = \nabla \cdot F = \nabla F.
\end{equation}

Since $\nabla \cdot F$ is a vector, $F$ must be a curl free bivector. This is precisely Maxwell's equation.

\section{Discussion}

Note that the usual Poincar\'e lemma $\nabla \cdot J = 0 \implies \exists F : J = \nabla \cdot F$ doesn't tell us that $J$ must be the divergence of a bivector, just that there exists such a bivector. In the same way, the ``strong'' Poincar\'e lemma presented here doesn't tell us that $F$ must satisfy Maxwell's equations, just that there exists such an $F$. In this view, the result is not particularly surprising.

Consider the more common situation of applying Poincar\'e's lemma to the electromagnetic field $F$. We use the fact that $\nabla \wedge F = 0$ to determine the existence of a potential $A$ satisfying $F = \nabla \wedge A$. Here, we regularly assume that we can choose the Lorenz gauge $\nabla \cdot A = 0$. The strengthened lemma simply tells us that there is no obstruction to making this choice.

In a certain sense, $\nabla \wedge F = 0$ is also a gauge choice, insofar as the current density $J$ is concerned. That is, adding a divergence free bivector to $F$ does not change the physical content of $J$, so in this sense it is like a gauge transformation with respect to $J$. However, it \emph{does} change the physical content of $F$, unlike gauge transformations of $A$, so in this sense it is not a gauge transformation, at least with respect to $F$.

On the other hand, this lemma means that Maxwell's equation is simply and precisely an expression of local charge conservation. It still carries the \emph{freedom} to admit magnetic sources, but Maxwell's equation with both electric and magnetic sources can always be decoupled into a pair of Maxwell equations: one for electric sources and one for magnetic sources. Of course, this says nothing of the dynamics of electromagnetism, since it tells us nothing of the force exerted by the field on the current.

But, it does tells us that the force law is what distinguishes different electromagnetic theories in trivial topologies, and helps us to know that topological theories of electrodynamics are fundamentally distinct, only resembling the classical theory locally. A general theory of conserved currents on manifolds rooted in de Rham theory is certainly worth exploring.

As such, this result is helpful in guiding investigations of extensions to electrodynamics. For instance, some equations may appear to be generalizations of Maxwell's equations, but are not. Consider $\nabla F + \nabla \chi = J$, where $\chi$ is a scalar field satisfying $\nabla^2 \chi = 0$, as seen in \cite{dvoe} with $J=0$. The strong lemma tells us that solutions to this equation have corresponding solutions to Maxwell's equations.

It also simplifies axiomatic treatments to electrodynamics, affirming the argument of Jos\'e Heras that charge conservation can be used as the fundamental assumption underlying Maxwell's equations.\cite{heras} 

Moreover, this result applies to any conserved current, so may offer insight into the construction of theories beyond electrodynamics. Corresponding to any locally conserved quantity (and thus, by Noether's theorem, to any symmetry) is a field satisfying Maxwell's equation. Perhaps with clever choice of a force law, Maxwell's equations can find use in the description other conserved quantities as well.

For instance, Itin, Hehl, and Obukhov develop a premetric theory of general relativity much like premetric electrodynamics, exploiting the analogies between the mathematical structures in electromagnetism and the teleparallel theory of gravity.\cite{itin} In viewing Maxwell's equations as the expression of a conserved quantity, it is less surprising that these analogies between electromagnetism and general relativity exist, and perhaps also less mysterious that Yang-Mills theories have structure so closely resembling that of electrodynamics. These analogies are not coincidental. They are simply the marks of conserved quantities.

Investigation of where these analogies between theories thrive and fail precisely will surely prove to be fruitful, as we scramble for grounding and hints at truthful theories beyond our current knowing of nature. As long as there are conserved quantities in the universe, Maxwell's equations will always be hiding in our theories --- at least locally.

\begin{thebibliography}{9}

  \bibitem{hehl}
    Fredriech W. Hehl, Yuri N. Obukhov.
    \emph{Foundations of Classical Electrodynamics.}
    Birkhäuser Basel (2003).

  \bibitem{cagc}
    D. Hestenes.
    \emph{Clifford algebra to geometric calculus}.
    D. Reidel Publishing Company (1984).
    
  \bibitem{gap}
      C. Doran and A. Lasenby.
      \emph{Geometric Algebra for Physicists}. Cambridge University Press (2003).

  \bibitem{griffiths}
      D. J. Griffiths.
      \emph{Introduction to Electrodynamics, 3rd edition.} 
      Prentice Hall (1999).       

  \bibitem{belinfante}
    F. J. Belinfante
    \emph{On the longitudinal and the transversal delta function, with some applications}.
    Physica, Volume 12, Issue 1 (1946).

  \bibitem{hestenes}
    D. Hestenes.
    \emph{Differential Forms in Geometric Calculus}.
    Clifford Algebras and their Applications in Mathematical Physics, 
    Kluwer: Dordercht/Boston(1993), 269–285.    

  \bibitem{brackx}
    F. Brackx, R. Delanghe, and F. Sommen.
    \emph{Differential Forms and/or Multi-vector Functions.}
    CUBO 7(2):139-169 (2005).    
    
  \bibitem{dvoe}
    Valeri V Dvoeglazov. 
    \emph{Generalized Maxwell equations from the Einstein postulate}.
    J. Phys. A: Math. Gen. 33 5011 (2000).
    
  \bibitem{heras}
    Jos\'e A. Heras.
    \emph{Can Maxwell’s equations be obtained from the continuity equation?}
    American Journal of Physics 75, 652 (2007).
    
  \bibitem{itin}
    Yakov Itin, Fredriech W. Hehl, Yuri N. Obukhov.
    \emph{Premetric equivalent of general relativity}.
    Phys. Rev. D 95, 084020 (2017).

\end{thebibliography}

\end{document}
