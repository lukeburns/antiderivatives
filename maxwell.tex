\documentclass{article}
\usepackage[margin=1in]{geometry}
\usepackage{amsmath}
\usepackage{hyperref}

\title{The Continuity Equation Implies Maxwell's Equations}
\author{Luke Burns}

\begin{document}
\maketitle
\abstract{It is shown that the antiderivative of a divergence free multivector (e.g. vector, bivector, spinor) field fails to be curl free by at most a (generalized) harmonic function. This result implies that \emph{any} vector valued current density $J$ that is divergence free possesses a bivector valued antiderivative $F$ that satisfies $\partial F = J$ under suitable boundary conditions. In four dimensions, this is Maxwell's equation. This reinforces an existing result and generalizes to conserved currents on manifolds of arbitrary dimensions. This not only indicates that charge conservation is itself sufficient for an axiomatic foundation for Maxwell's equations; by means of Noether's theorem, this result associates to every symmetry in four dimensions a bivector field satisfying Maxwell's equations, up to specific units.}

\paragraph{Work In Progress} This paper is a work in progress, and is being openly developed on Github at \url{https://github.com/lukeburns/maxwells-equations}. Contributions are warmly welcomed, whether by means of opening an issue or pull request.

\section{Introduction}

The question \emph{Can Maxwell’s equations be obtained from
the continuity equation?} was first asked by Jos\'e A. Heras in \cite{heras}, who concluded \emph{yes} and provided a construction by means of a generalization of Helmholtz decomposition. In \cite{heras-axiomatic}, Heras presents an axiomatic approach to electrodynamics beginning with the continuity equation. The purpose of the current paper is to affirm and generalize the result of Heras to manifolds of any dimension by means of a generalized Helmholtz decomposition of geometric calculus, as well as to discuss its implications beyond electrodynamics.

After establishing a mapping between $n$-vector fields and differential forms of degree $n$, which allows for the result of this paper to be translated directly into differential forms, I present two key results of geometric calculus: a generalized Integral Formula and Helmholtz decomposition for fields. 

Using these results, I show that all divergence free fields fail to be coexact (i.e. the curl of some other field) by at most a monogenic term. Monogenic fields are characterized by the property that they are fully determined by boundary conditions, analogous to complex analytic functions. I present some conditions under which these fields are coexact. A field whose antiderivative is curl free is dubbed \emph{faithful}, by which it follows that the derivative of a curl free field is faithful, and the antiderivative of a faithful field is curl free. This establishes an equivalence between the statements ``an electromagnetic field $F$ is a curl free bivector field'' and ``an electromagnetic current $J$ is a faithful vector field,'' both of which fully determine the structure of Maxwell's equations. 

I then show that a conserved vector field $J$ is faithful on a simple manifold of arbitrary dimension under suitable boundary conditions. Not only does this reinforce the result of Heras that the continuity equation implies Maxwell's equations; it points to an intimate connection between symmetry and bivector fields like Maxwell's. Noether's theorem associates to every symmetry a conserved current. In this paper, an explicit expression is given that associates to every conserved current a curl free bivector field that is its antiderivative. This means that to every symmetry there is an associated bivector field that is identical to an electromagnetic field, up to its units.

\section{Fields and forms} If $F_n \equiv \langle F \rangle_n$ is the grade $n$ part of the multivector field (hereafter, just field) $F = \sum F_n$ in an arbitrary geometric algebra $\mathcal{G}$, then its corresponding \emph{differential form $f_n$ of degree $n$} is a scalar field given by \cite{cagc}

\begin{equation}
  f_n \equiv d^nx^\dagger \cdot F_n,\label{eq:form}
\end{equation}

which is the projection of the $n$-vector field $F_n$ onto the directed measure $d^nx^\dagger = dx_n \wedge \dots \wedge dx_1$, where $dx_i$ are vector valued differentials. 

The hodge star operation $*$ acts on fields as

\begin{equation}
  *F \equiv F^\dagger I,\label{eq:hodge}
\end{equation}

where $I$ is the pseudoscalar of some oriented vector manifold.

The exterior derivative $d$ behaves identically to the curl

\begin{equation}
  d f_n \equiv d^{n+1}x^\dagger \cdot (\partial \wedge F_n),\label{eq:curl}
\end{equation}

and the ``adjoint operator'' $\delta$ behaves identically to (minus) the divergence

\begin{equation}
  \delta f_n \equiv d^{n-1}x^\dagger \cdot (-\partial \cdot F_n).\label{eq:divergence}
\end{equation} 

The word \emph{form} will be reserved for scalar fields corresponding to some $n$-vector field via Equation \ref{eq:form}. Lowercase letters will be used for forms and uppercase letters for fields. Subscripts denote grade of a multivector (degree of a form).

\section{Derivatives}

A field $F$ is called \emph{curl free} (or, closed) when 

\begin{equation}
  \partial \wedge F = 0
\end{equation}

and \emph{divergence free} (or, coclosed) when 

\begin{equation}
  \partial \cdot F = 0,
\end{equation}

where $\partial = \partial_x$ is the derivative with respect to the vector $x$. This operator is unique to geometric calculus, and the entire subject is a study of the properties of this operator. It's also often called the Dirac operator and can be written $\partial = e^k \partial_k$ where $\partial_k = \frac{\partial}{\partial x^k} = e_k \cdot \partial$ with respect to coordinates $x^k = e^k \cdot x$.\footnote{See p. 252 of \cite{cagc} for a coordinate free, integral definition of $\partial$.}

A field with no divergence or curl

\begin{equation}
  \partial F = \partial \cdot F + \partial \wedge F = 0
\end{equation}

is called \emph{monogenic}. It possesses the property of complex analytic functions that, in any region, it is fully determined by its values on the boundary of that region. Hence, the form $\omega$ is closed if $d \omega = 0$, coclosed if $\delta \omega = 0$, and monogenic if $d \omega = \delta \omega = 0$.

A field $H$ that satisfies

\begin{equation}
  \partial^2 H = 0
\end{equation}

might be called \emph{harmonic}, although the term is inappropriate in mixed signature spaces. For instance, in Minkowski space, $\partial^2 H = (\partial_t^2 - \vec \partial^2) H = 0$ is the wave equation and its properties differ dramatically from the usual harmonic functions in Euclidean spaces. Nonetheless, we will abuse the term here for lack of a better one. A form $\gamma$ is then harmonic if $d \delta \gamma + \delta d \gamma = 0$.

\section{Potentials}

If a field $J$ is written as

\begin{equation}
  J = \partial \cdot G + \partial \wedge H,
\end{equation}

then $G$ and $H$ are called \emph{potentials} for $J$. By extension, if a form $\omega$ is given by

\begin{equation}
  \omega = d \alpha + \delta \beta
\end{equation}

then $\alpha$ and $\beta$ will be called potentials for $\omega$.

A field $J$ is called \emph{exact} when

\begin{equation}
  J = \partial \wedge F
\end{equation}

and \emph{coexact} when

\begin{equation}
  J = \partial \cdot F,
\end{equation}

whereby a form $\omega$ is exact if $\omega = d \alpha$ and coexact if $\omega = \delta \beta$. Two curl free fields are called \emph{cohomologous} if their difference is an exact field. It should now be obvious that two divergence free fields are called \emph{cocohomologous} if their difference is a coexact field.\footnote{This is made up.}

\section{Antiderivatives}

A field $F$ is called an \emph{antiderivative} of $J$ if

\begin{equation}
  J = \partial F = \partial \cdot F + \partial \wedge F,\label{eq:antiderivative}
\end{equation}

which is unique up to a monogenic term. That is, $F + C$ such that $\partial C = 0$ is also an antiderivative. Furthermore, given an antiderivative, one has possession of constraints on $F$. For every $J_k = 0$,

\begin{equation}
  J_k = \partial \cdot F_{k+1} + \partial \wedge F_{k-1} = 0.\label{eq:constraints}
\end{equation}

As an example, if $J = J_n$ is an $n$-vector field, then 

\begin{equation}
  J_n = \partial F = \partial \cdot F_{n+1} + \partial \wedge F_{n-1},\label{ex:antiderivative}
\end{equation}

and the constraints due to $J_{n-1} = J_{n+1} = 0$ are

\begin{equation}
  \partial \cdot F_{n-1} = \partial \wedge F_{n+1} = 0.\label{ex:constraints}
\end{equation}

Of course, $F$ \emph{could} contain terms of higher and lower grades, but they make no contribution to $J_n$. In this case, it will generally be of the form $F = F_{n-1} + F_{n+1} + C$.

If $j_n$ and $f_n$ are the forms given by $J_n$ and $F_n$, then Equation \ref{ex:antiderivative} is equivalent to

\begin{equation}
  j_n = \delta f_{n+1} + d f_{n-1},
\end{equation}

and Equation \ref{ex:constraints} is equivalent to

\begin{equation}
  \delta f_{n-1} = d f_{n+1} = 0.
\end{equation}

Given potentials $f_{n-1}$ and $f_{n+1}$ under these constraints, one is in possession of an antiderivative of $j_n$.

\section{The Fundamental Theorem}

Let $\mathcal{M}$ be an m-dimensional smooth oriented vector manifold\footnote{See Chapter 4 of \cite{cagc}, or Section 6.5 of \cite{gap}.} with a piecewise smooth boundary $\partial \mathcal{M}$ and $L$ be a linear function, differentiable on $\mathcal{M}$ and $\partial \mathcal{M}$. Then, \cite{cagc} \cite{sobczyk} \cite{gap}

\begin{equation}
  \int L(\dot x, d^mx \dot \partial) = \oint L(x, d^{m-1}x),\label{eq:fundamental-theorem}
\end{equation}

where $L(\dot x, d^mx \dot \partial)$ denotes right and left differentiation all $x$ dependent terms in $L$ by $\partial$.

Stokes' theorem of differential forms is

\begin{equation}
  \int \langle L(\dot x, d^mx \dot \partial)\rangle = \oint \langle L(x, d^{m-1}x)\rangle,
\end{equation}

for scalar valued integrands.

\section{Integral Formula}

Let $J$ be a field on a simple (not self-intersecting) manifold $\mathcal{M}$ subject to the same criteria in the fundamental theorem. Suppose $J$ satisfies the equation

\begin{equation}
  \partial F = J.
\end{equation}

Then $F$ is given by \cite{cagc}

\begin{equation}
  F(x) = (-1)^m I^{-1}(x) \left(\int g(x, x') d^{m}x' J(x') - \oint g(x, x') d^{m-1}x' F(x')\right),\label{eq:integral-formula}
\end{equation}

where $g$ is a Green's function of $\partial$ satisfying $\partial g(x,x') = - g(x, x') \partial' = \delta(x - x').$ This result says that \emph{any integrable field has an antiderivative, and it's given by Equation \ref{eq:integral-formula}}.

\section{Helmholtz decomposition}

The integral forumla tells us that $J$ has an antiderivative $F$ such that

\begin{align}
  J = \partial F = \partial \cdot F + \partial \wedge F &= (-1)^m I^{-1} \left(\int g d^{m}x \partial^2 F - \oint g d^{m-1}x \partial F\right).\label{eq:helmholtz}
\end{align}

In addition, we can say

\begin{equation}
  \partial \cdot F = (-1)^m I^{-1} \left(\int g d^{m}x \partial (\partial \cdot F) - \oint g d^{m-1}x \partial \cdot F\right)
\end{equation}

and

\begin{equation}
  \partial \wedge F = (-1)^m I^{-1} \left(\int g d^{m}x \partial (\partial \wedge F) - \oint g d^{m-1}x \partial \wedge F\right),
\end{equation}

which gives a generalized Helmholtz decomposition into divergence free (or coclosed) and curl free (or closed) fields, $\partial \cdot F$ and $\partial \wedge F$ respectively. This is because $\partial \wedge (\partial \wedge M) = \partial \cdot (\partial \cdot M) = 0$ for any field $M$. 

Additionally, this decomposition comes with constraints given by Equation \ref{eq:constraints}.

\section{Antiderivatives of divergence free fields} The above result implies that antiderivatives of divergence free fields fail to be curl free, and antiderivatives of curl free fields fail to be divergence free, by at most a harmonic function $H$ satisfying $\partial^2 H = 0$.

Suppose $J = \partial F$ is divergence free (the dual result for curl free fields follows analogously). Then, 

\begin{equation}
  \partial \cdot J = \partial \cdot (\partial F) = \partial \cdot (\partial \wedge F) = \partial (\partial \wedge F) = 0,
\end{equation}

which means that $C \equiv \partial \wedge F$ is monogenic and $J$ is \emph{cohomologous} with $C$

\begin{equation}
  J - C = \partial F - C = \partial \cdot F,\label{eq:cohomologous}
\end{equation}

because their difference is coexact.

Employing the integral theorem, $C$ has an antiderivative $H$ such that 

\begin{equation}
  C = \partial H.
\end{equation}

With $G \equiv F - H$, this implies that $F$ can then be written

\begin{equation}
  F = G + H\label{eq:harmonic-split}
\end{equation}

where $\partial G = \partial \cdot F$ and $\partial^2 H = 0$. Hence, $F$ fails to be curll free by at most a harmonic function $H$.

As an example, if $C$ is an r-vector field, then $C$ can be written $C = \partial \cdot (x \wedge C)/r = \partial \wedge (x \cdot C)/(n-r)$, and $C$ is both exact and coexact, in which case\footnote{Under what conditions are monogenic fields (co)exact?}

\begin{equation}
  J = \partial F = \partial \cdot (F + x \wedge C/r)
\end{equation}

is coexact --- although, $F$ is not curl free. 

If $C = 0$ on the boundary, then $C = 0$ everywhere, and its antiderivative is curl free

\begin{equation}
  J = \partial F = \partial \cdot F. \label{eq:faithful}
\end{equation}

Let us call a field $J$ \emph{faithful} if its antiderivative is curl free. Faithful fields are coexact, and all divergence free fields differ from faithful fields by at most a monogenic field, which depends solely on the manifold and boundary conditions. Note, however, that coexact fields are not necessarily faithful. Of course, we might choose to call a field the antiderivative that is divergence free \emph{cofaithful}, though that admittedly doesn't quite have the same ring to it.\footnote{Might be better to abandon the word faithful altogether for something more descriptive.}

\section{Maxwell's equations}

Maxwell's equations follow from directly from the statement that \emph{``an electromagnetic field $F$ is a curl free bivector field and its derivative is its current $J$.''} To say that it is curl free means that

\begin{equation}
  \partial \wedge F = 0,
\end{equation}

and its derivative $J$ is

\begin{equation}
  J = \partial F = \partial \cdot F,
\end{equation} 

which are Maxwell's equations with no magnetic monopoles or currents. Of course, Maxwell's equations imply the continuity equation

\begin{equation}
  \partial \cdot J = \partial \cdot (\partial \cdot F) = 0, \label{eq:continuity}
\end{equation} 

which means charge is conserved. 

$F$ is the antiderivative of $J$, so to say that $F$ is curl free is the same as saying $J$ is faithful. Hence, the above is equivalent to a dual statement \emph{``an electromagnetic current $J$ is a faithful vector field and its antiderivative is its electromagnetic field $F$.''}

Suppose $J$ is a conserved current. Then by Equation \ref{ex:constraints} and Equation \ref{eq:harmonic-split}, its antiderivative $F$ can be decomposed into 

\begin{equation}
  F = F_0 + F_2,
\end{equation}

where $F_0$ is harmonic.

Equation \ref{eq:helmholtz} tells us that, in four dimensions, $J$ can be written

\begin{align}
  J = \partial F &= I^{-1} \left(\int g d^{4}x \partial^2 F - \oint g d^{3}x J \right).
\end{align}

If we integrate over all of spacetime, the boundary term vanishes for suitable charge distributions, and $J$ is given by

\begin{align}
  J = \partial F &= I^{-1} \int g d^{4}x \partial^2 F,
\end{align}

which is only dependent on $\partial^2 F = \partial^2 (F_0 + F_2) = \partial^2 F_2$, since $F_0$ is harmonic. Hence, $J$ is independent of $F_0$, and $J$ can at last be written as 

\begin{equation}
  J = \partial F = \partial \cdot F,
\end{equation}

and $J$ is faithful.

\section{Symmetry}

Noether's theorem associatese to every symmetry a current $J$ satisfying $\partial \cdot J = 0$. Equation \ref{eq:integral-formula} provides a direct expression for the antiderivative $F$ of $J$ such that $\partial F = J$, which is a bivector satisfying Maxwell's equations (under suitable boundary conditions). \emph{What are the implications of this?}

  \begin{thebibliography}{9} 

    \bibitem{heras}
      J. Heras.
      \emph{Can Maxwell’s equations be obtained from the continuity equation?}.
      Am. J. Phys. Vol. 75, No. 7. (2007).
      \url{https://arxiv.org/pdf/0812.4785.pdf}

    \bibitem{heras-axiomatic}
      J. Heras.
      \emph{An axiomatic approach to Maxwell’s equations}.
      Eur. J. Phys. Vol. 37, No. 5. (2016).
      \url{https://arxiv.org/pdf/1608.00659.pdf}

    \bibitem{cagc}
      D. Hestenes.
      \emph{Clifford algebra to geometric calculus}.
      D. Reidel Publishing Company (1984).

    \bibitem{sobczyk}
    Sobczyk, G. \& Sánchez.
    \emph{Fundamental Theorem of Calculus}.
    O.L. Adv. Appl. Clifford Algebras Vol. 21, No. 221 (2011).
    \url{https://arxiv.org/pdf/0809.4526.pdf}

    \bibitem{gap} 
      C. Doran and A. Lasenby.
      \emph{Geometric Algebra for Physicists}. Cambridge University Press (2003).

  \end{thebibliography}

\end{document}
